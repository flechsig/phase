Optimierung

Die Optimierung kann nur in der Naehe des lokalen Minimums der 
Costfunktion (gegenwaertig nur noch Quadrat der Defokussierung, d.h. vertikale
Ablage in der Bildebene) zum Erfolg fuehren. Die Startwerte sind entsprechend 
zu waehlen. Sind gute Startwerte nicht bekannt, sollte man zuerst mit rel.
grossem Fehler und ohne Grenzen optimieren. Laeft die Optimierung in
"unphysikalische" Bereiche muessen Grenzen eingefuehrt werden. Bei
erfolgreicher Optimierung kann der Fehler verkleinert und mit neuem Startwert
erneut optimiert werden. Der ausgegebene Wert der Costfunktion muss
waehrend der Optimierung kleiner werden, bei sehr grossen Werten muss der 
Wichtungsfaktor in der Costfunktion angepasst werden (gegenwaertig nur von mir
mit neu- Compilierung moeglich).
Bugs: Die "der Inhalt der Opti Box" muss kritisch im Auge behalten werden. Hat
er sich unerklaerlicherweise geaendert-- tritt auf nach Absturz des
Optimierungsprogramms in sinnlose Bereiche, sollte man nicht mit "OK" den
Inhalt speichern, sondern die Box schliessen (CANCEL oder mit Maus) und im
FILE, FILES- Menue mit OK die Dateien erneut laden.
