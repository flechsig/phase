%  File      : /afs/psi.ch/user/f/flechsig/phase/doc/install.tex
%  Date      : <28 Oct 03 11:01:17 flechsig> 
%  Time-stamp: <28 Oct 03 11:03:57 flechsig> 
%  Author    : Uwe Flechsig, flechsig@psi.ch

%  $Source$ 
%  $Date$
%  $Revision$ 
%  $Author$ 

% Datei: USERDISK_3:[FLECHSIG.PHASE.TEXT]INSTALL.TEX
% Datum: 22.JUN.1995
% Stand: 18-MAR-1996
% Autor: FLECHSIG, BESSY Berlin


\chapter{Introduction}
\phase is a program package for calculations at optical systems used with 
syncrotron radiation. The main features are:
\begin {itemize}
  \item complete phase space transformation 
  \item ray tracing
  \item beamline optimization
  \item calculation of optical abberations for complete systems
  \item phase sensitive calculation
  \item input of undulator source date from \pname {Wave}
  \item X- window API for input and graphical output
\end {itemize}

The package consists of independent calculation programs and a user
application programming interface (API) for the Motiv (X-) window manager.  
With the interface it is easy to generate the input files for the
calculations. The calculation programs can be started from the API or
seperately, for example in a batch job. 

The calculation prinziple and basic ideas of the program are explained
in \cite{BAHRDT95a} and \cite{BAHRDT95b}.

\chapter{Installation}
The \phase program at BESSY is availible at \prog{prg\_root:[phase]}.
The installation is very simple using the installation makefile. After
installation the file system is generated and PHASE can be used on
your system.  Later you have to set up the appropriate environment
starting the setup program (\prog{phaseinit.com}) at the begining of
the session. That program and some other files should be found after
installation in your PHASE directory- default: \~/phase. PHASE can be
removed as simple as installed: \prog {make clear} \footnote {The
files HIGZ\_WINDOWS.DAT and the makefile itselve are not removed
automatically!}.

\section{Default Installation}
At your account the directory ~/phase is generated and some files are
copied.  For individual installation read the makefile. If there is no
\~/phase directory on your system no problems should arise. For PAW-
users: a X- defaults file for HIGZ (position and size of the HIGZ-
window) will be copied into your home directory, if you have your own
setting- remove the copied version.  It is recommended to do as
follows:
\begin{enumerate}
	\item Copy the makefile to any directory on your account. The
   command: (at BESSY) \prog {copy prg\_root:[phase]makefile *} will
   copy the makefile to your current directory.  
	\item run the
   makefile \prog{make install}.
\end{enumerate}



\section{PHASE at AXP- Systems}
The \phase program runs at VAX-- and AXP-- (Alpha) systems. The setup
program \prog {phaseinit.com} detects your system and sets the right
symbols automatically. The makefiles for optimization/extraction do it
as well.

\tiny{\it filename: /afs/psi.ch/user/f/flechsig/phase/doc/install.tex}  
